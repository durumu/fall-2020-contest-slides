\section{G}%
\label{sec:g}

\begin{frame}
  \frametitle{Problem G: Size Constraints}

  \begin{itemize}
    \item First solve at 00:38 by \textbf{backslash7} (Matthew Davis, Nicholas Glaze, and Travis Baylor)
    \item We want to change the minimum number of elements in $a$ such that $b$ never occurs as a continuous subsegment
    \item \textbf{Insight:} since there are so many numbers that are allowed as sizes, we can get rid of any occurrence
    \item Let's go left to right until we find the first occurrence of $b$. Clearly, we must change something.
      \begin{itemize}
        \item We can greedily choose to change the last element of the occurrence. It still gets rid of this occurrence, and ``has the most potential''
          to remove future occurrences
      \end{itemize}
    \item \textbf{Potential pitfall:} $b$ can be longer than $a$, in which case the answer is always 0
  \end{itemize}
\end{frame}

\begin{frame}
  \frametitle{Problem G: Python Solution}
  \pyf{G}
\end{frame}
