\documentclass{beamer}
\usepackage{listings}
\usepackage{textcomp}
\usepackage{amsmath}

% Listings
    \definecolor{dkgreen}{rgb}{0,0.6,0}
    \definecolor{gray}{rgb}{0.5,0.5,0.5}
    \definecolor{mauve}{rgb}{0.58,0,0.82}

    \lstloadlanguages{[ISO]C++, Python}

    \lstset{frame=tb,
      upquote=true,
      language=[ISO]C++,
      aboveskip=2mm,
      belowskip=2mm,
      showstringspaces=false,
      columns=flexible,
      basicstyle={\small\ttfamily},
      numbers=none, % line numbers -- change to "left" if you want them
      numberstyle=\tiny\color{gray},
      keywordstyle=\bfseries\color{blue},
      commentstyle=\itshape\color{dkgreen},
      stringstyle=\color{mauve},
      breaklines=true,
      breakatwhitespace=true,
      tabsize=2
    }

    % C++ environment
    \lstnewenvironment{cpp} {
      \lstset{
        language=[ISO]C++,
        tabsize=2
      }
    } {}

    % Python environment
    \lstnewenvironment{py} {
      \lstset{
        language=Python,
        tabsize=4
        }
    } {}

\newcommand{\cppf}[1]{\lstinputlisting{#1.cpp}}
\newcommand{\pyf}[1]{\lstinputlisting[language=Python, upquote=true]{#1.py}}

\usetheme{Madrid}

\title{Editorial}
\subtitle{Texas A\&M 2020 Spring Contest}
\date{\today}

\begin{document}

\section{A}%
\label{sec:a}

\begin{frame}
  \frametitle{Problem A: Are You Ready?}

  \begin{itemize}
    \item First solve at 00:00 by \textbf{alab11} (Alex Labbane)
    \item A little introduction to today's theme!
  \end{itemize}

  \pyf{A}
\end{frame}
 % 0:00
\section{B}%
\label{sec:B}

\begin{frame}
  \frametitle{Problem B: Build a Snowman}
  \begin{itemize}
    \item First solve at 00:07 by \textbf{i\_am\_a\_business\_major} (Logan Baker)
    \item \textbf{Observation}: any values in between the first and the last values in the build array does not contribute to the sturdiness factor at all.\\
    $\frac{a_2}{a_1} * \frac{a_3}{a_2} * \frac{a_4}{a_3} * ... * \frac{a_{n-1}}{a_{n-2}} * \frac{a_n}{a_{n-1}} = \frac{a_n}{a_1}$
    \item If we choose the maximum radius as $a_n$ and the minimum radius as $a_1$, the sturdiness factor will be maximized.
    \item \textbf{Potential pitfall}: The answer has to be precise to $10^-6$. Your answer could be considered wrong if you don't print enough decimal digits. For example: $2.62$ is interpreted as $2.6200000$ while the correct answer can be $2.6200011$.
  \end{itemize}
\end{frame}


\begin{frame}
	\frametitle{Problem B: C++ solution}
	\cppf{B}
\end{frame}
 % 0:07
\section{D}%
\label{sec:d}

\begin{frame}
  \frametitle{Problem D: Dubious Logs}

  \begin{itemize}
    \item First solve at 00:22 by \textbf{whatever} (Chinh Luu and Matthew Tran)
    \item We can reduce the problem to the following: given a list $a$ and a list $l$,
      match each element of $l$ to an element of $a$ that is less than or equal to it.
    \item \textbf{Observation:} it is never worse to assign an element of $l$ closest
      element of $a$ less than or equal to it that we have not already assigned, so we
      can just sort both lists and try this assignment.
  \end{itemize}
  \pyf{D}
\end{frame}
 % 0:17
\section{G}%
\label{sec:g}

\begin{frame}
  \frametitle{Problem G: Size Constraints}

  \begin{itemize}
    \item First solve at 00:38 by \textbf{backslash7} (Matthew Davis, Nicholas Glaze, and Travis Baylor)
    \item We want to change the minimum number of elements in $a$ such that $b$ never occurs as a continuous subsegment
    \item \textbf{Insight:} since there are so many numbers that are allowed as sizes, we can get rid of any occurrence
    \item Let's go left to right until we find the first occurrence of $b$. Clearly, we must change something.
      \begin{itemize}
        \item We can greedily choose to change the last element of the occurrence. It still gets rid of this occurrence, and ``has the most potential''
          to remove future occurrences
      \end{itemize}
    \item \textbf{Potential pitfall:} $b$ can be longer than $a$, in which case the answer is always 0
  \end{itemize}
\end{frame}

\begin{frame}
  \frametitle{Problem G: Python Solution}
  \pyf{G}
\end{frame}
 % 0:17
\section{C}%
\label{sec:c}

\begin{frame}
  \frametitle{Problem C: Cold War}

  \begin{itemize}
    \item First solve at 00:27 by \textbf{Jonathan1234} (Naixin Zong)
    \item Implementation problem: figure out if you can make it out of a duel without getting shot
    \item One possible solution: try shooting everyone and see what happens 
    \item Complexity: $O(n^2)$
    \item Another possible solution: order everyone by reaction time, then step through until your turn
    \item Complexity: $O(n \log n)$
  \end{itemize}

\end{frame}

\begin{frame}
  \frametitle{Problem C: $O(n^2)$ try-everyone solution}
  \pyf{Cbrute}
\end{frame}

\begin{frame}
  \frametitle{Problem C: $O(n^2)$ try-everyone solution (continued)}
  \pyf{Cbrute2}
\end{frame}

\begin{frame}
  \frametitle{Problem C: $O(n \log n)$ ordering solution}
  \pyf{Cfast}
\end{frame}
 % 0:27
\section{H}%
\label{sec:h}

\begin{frame}
  \frametitle{Problem H: Help the Carolers}

  \begin{itemize}
    \item First solve at 00:38 by \textbf{Name Here}
    \item Determine if there is a word or group of words present in both the end of the first phrase and start of the second
    \item Iterate backwards over first phrase checking for equality
    \item \textbf{Potential pitfall}: Case where the second phrase is non-ending substring of the first
    \item Total complexity: $O(n^2)$
  \end{itemize}
\end{frame}

\begin{frame}
  \frametitle{Problem H: Python Solution}
  \pyf{H}
\end{frame}
 % 0:42
\section{F}%
\label{sec:f}

\begin{frame}
  \frametitle{Problem F: Flummoxing Map}

  \begin{itemize}
    \item First solve at 00:58 by \textbf{0e4ef622} (Matthew Tran)
    \item Tricky problem, but lots of ways to solve it if you observe one of 2 things
    \item \textbf{Observation}: There are only 8 possible maps!
    \item \textbf{Observation}: This is just a linear transformation on the coordinates!
  \end{itemize}

\end{frame}

\begin{frame}
  \frametitle{Problem F: 8 Maps}
  \begin{itemize}
    \item We can uniquely identify a map by a rotation and whether it is mirrored
    \item Can think of this as a graph where the transformations are edges and there are 8 vertices
    \item You can convince yourself of this by trying various combinations of rotations and flips
    \item Or if you're into math you can prove it with group theory, I guess
    \item You can explicitly generate these maps, or just transform the coordinates depending on state
  \end{itemize}
\end{frame}

\begin{frame}
  \frametitle{Problem F: 8 Maps Solution}
  \pyf{F}
\end{frame}

\begin{frame}
  \frametitle{Problem F: Linear Transformation}
  \begin{itemize}
    \item If input coordinates are $i$, $j$, then mirroring changes them to $i$, $n-j+1$
    \item Rotating CCW replaces them with $j$, $n-i+1$
    \item Similar situation for flipping and rotating CW
    \item These are actually just 2x2 matrices
    \item Can just multiply a long string of 2x2 matrices to get our result
  \end{itemize}
\end{frame}

\begin{frame}
  \frametitle{Problem F: Linear Transformation Solution}
  \cppf{F}
\end{frame}
 % 0:58
\section{E}%
\label{sec:e}

\begin{frame}
  \frametitle{Problem E: Chili Cook-off}
  \begin{itemize}
    \item First solve at 01:01 by \textbf{ti123} (Hang Li)
    \item The event space consists of every possible combination of trucks starting. There are $2^n$, since each truck either starts or doesn't
    \item Notice that $n \le 10$, so we can use brute force
    \item You can use recursion, but an easier way is to recognize that each bitmask of length $n$
      corresponds to a unique event, where person $i$ has a good truck day if bit $i$ is set
  \end{itemize}
\end{frame}
\begin{frame}
  \begin{itemize}
    \item So let's loop over all bitmasks of length $n$, and for each, we can calculate the probability of this event as \[ \left( \prod_{\text{bit $i$ is 1}} p[i] \right) \left( \prod_{\text{bit $i$ is 0}} (1-p[i]) \right). \]
    \item Once we have the probability, we determine what place we would get if this event occurs, and add this probability to the output for that place
    \item Bonus: can you solve it in $O(n^2)$?
  \end{itemize}
\end{frame}

\begin{frame}
  \frametitle{Problem E: C++ Solution}
  \cppf{E}
\end{frame}
\begin{frame}
  \frametitle{Problem E: C++ Solution (continued)}
  \cppf{E2}
\end{frame}
 % 1:11
\section{I}%
\label{sec:i}

\begin{frame}
  \frametitle{Problem I: Square Weed Whacker}

  \begin{itemize}
    \item First solve at 00:35 by \textbf{whatever} (Chinh Luu and Matthew Tran)
    \item We want the largest square such that every \texttt{.} can be covered without the square touching a \texttt{\#}
    \item One option is to binary search on square side length and use a prefix sum structure
    \item A cleaner approach uses a really nice insight: the side length is upper bounded by the shortest vertical or horizontal sequence of $\texttt{.}$
    \item A quick proof by contradiction (or a proof by AC) shows that we can attain this upper bound:
      \begin{itemize}
        \item Let $x$ be the shortest contiguous sequence of vertical or horizontal \texttt{.}
        \item Clearly, $x$ is an upper bound for the square's side, as otherwise we would definitely go out of bounds or hit a \texttt{\#}
        \item It remains to show that we can attain $x$. Indeed, if we couldn't, there must be a shorter sequence of \texttt{.} than $x$. But this is a contradiction!
      \end{itemize}
  \end{itemize}
\end{frame}

\begin{frame}
  \frametitle{Problem I: Python Solution}
  \pyf{I}
\end{frame} % :(
\section{J}%
\label{sec:j}

\begin{frame}
  \frametitle{Problem J: History Lesson}

  \begin{itemize}
    \item First solve at 00:17 by \textbf{whatever} (Chinh Luu and Matthew Tran)
    \item Straightforward implementation problem
    \item Just use a hashmap mapping strings to lists
    \item \texttt{unordered\_map<string, vector<string>>} in C++
    \item \texttt{collections.defaultdict(list)} in Python
    \item Some teams struggled with sorting the list alphabetically
  \end{itemize}
\end{frame}

\begin{frame}
  \frametitle{Problem J: Solution}
  \pyf{J}
\end{frame}
 % :(
  
\end{document}
