\documentclass{beamer}
\usepackage{listings}
\usepackage{textcomp}
\usepackage{amsmath}

% Listings
    \definecolor{dkgreen}{rgb}{0,0.6,0}
    \definecolor{gray}{rgb}{0.5,0.5,0.5}
    \definecolor{mauve}{rgb}{0.58,0,0.82}

    \lstloadlanguages{[ISO]C++, Python}

    \lstset{frame=tb,
      upquote=true,
      language=[ISO]C++,
      aboveskip=2mm,
      belowskip=2mm,
      showstringspaces=false,
      columns=flexible,
      basicstyle={\small\ttfamily},
      numbers=none, % line numbers -- change to "left" if you want them
      numberstyle=\tiny\color{gray},
      keywordstyle=\bfseries\color{blue},
      commentstyle=\itshape\color{dkgreen},
      stringstyle=\color{mauve},
      breaklines=true,
      breakatwhitespace=true,
      tabsize=2
    }

    % C++ environment
    \lstnewenvironment{cpp} {
      \lstset{
        language=[ISO]C++,
        tabsize=2
      }
    } {}

    % Python environment
    \lstnewenvironment{py} {
      \lstset{
        language=Python,
        tabsize=4
        }
    } {}

\newcommand{\cppf}[1]{\lstinputlisting{#1.cpp}}
\newcommand{\pyf}[1]{\lstinputlisting[language=Python, upquote=true]{#1.py}}

\usetheme{Madrid}

\title{Editorial}
\subtitle{Texas A\&M 2020 Spring Contest}
\date{\today}

\begin{document}

\section{A}%
\label{sec:a}

\begin{frame}
  \frametitle{Problem A: Howdy World}

  \begin{itemize}
    \item First solve at 00:01 by \textbf{whatever} (Chinh Luu and Matthew Tran)
    \item ``Hello World'' with a TAMU flair
  \end{itemize}

  \pyf{A}
\end{frame} % 0:00
\section{B}%
\label{sec:b}

\begin{frame}
  \frametitle{Problem B: License Plates}

  \begin{itemize}
    \item First solve at 00:08 by \textbf{whatever} (Chinh Luu and Matthew Tran)
    \item Straightforward implementation
    \item \textbf{Potential pitfall}: no vowels or the letter Q
    \item Not represented in sample input, so a lot of WAs from this
  \end{itemize}
\end{frame}

\begin{frame}
  \frametitle{Problem B: Python Solution}
  \pyf{B}
\end{frame}
 % 0:07
\section{D}%
\label{sec:d}

\begin{frame}
  \frametitle{Problem D: Fence Posts}

  \begin{itemize}
    \item First solve at 00:22 by \textbf{whatever} (Chinh Luu and Matthew Tran)
    \item Formally, we want a sequence such that the longest increasing subsequence is of length $a$, but the greedy algorithm described gives an increasing subsequence of length $b$
    \item You may observe that the greedy algorithm always takes the first fence post. Let's exploit this.
    \item It turns out that we can come up with a sequence of length $a+1$ which works on general. Given $a$ and $b$: $a-b+2$, $1$, $2$, $3$, $\dots$, $a$.
    \item For example, if $a=10$, $b=5$, we get $7$, $1$, $2$, $3$, $4$, $5$, $6$, $7$, $8$, $9$, $10$.
    \item \textbf{Potential pitfall:} there are many constructions, but if yours uses more than 150 numbers you will get WA
  \end{itemize}
  \pyf{D}
\end{frame}
 % 0:17
\section{G}%
\label{sec:g}

\begin{frame}
  \frametitle{Problem G: Size Constraints}

  \begin{itemize}
    \item First solve at 00:38 by \textbf{backslash7} (Matthew Davis, Nicholas Glaze, and Travis Baylor)
    \item We want to change the minimum number of elements in $a$ such that $b$ never occurs as a continuous subsegment
    \item \textbf{Insight:} since there are so many numbers that are allowed as sizes, we can get rid of any occurrence
    \item Let's go left to right until we find the first occurrence of $b$. Clearly, we must change something.
      \begin{itemize}
        \item We can greedily choose to change the last element of the occurrence. It still gets rid of this occurrence, and ``has the most potential''
          to remove future occurrences
      \end{itemize}
    \item \textbf{Potential pitfall:} $b$ can be longer than $a$, in which case the answer is always 0
  \end{itemize}
\end{frame}

\begin{frame}
  \frametitle{Problem G: Python Solution}
  \pyf{G}
\end{frame}
 % 0:17
\section{C}%
\label{sec:c}

\begin{frame}
  \frametitle{Problem C: Cold War}

  \begin{itemize}
    \item First solve at 00:27 by \textbf{Jonathan1234} (Naixin Zong)
    \item Implementation problem: figure out if you can make it out of a duel without getting shot
    \item One possible solution: try shooting everyone and see what happens 
    \item Complexity: $O(n^2)$
    \item Another possible solution: order everyone by reaction time, then step through until your turn
    \item Complexity: $O(n \log n)$
  \end{itemize}

\end{frame}

\begin{frame}
  \frametitle{Problem C: $O(n^2)$ try-everyone solution}
  \pyf{Cbrute}
\end{frame}

\begin{frame}
  \frametitle{Problem C: $O(n \log n)$ ordering solution}
  \pyf{Cfast}
\end{frame}
 % 0:27
\section{H}%
\label{sec:h}

\begin{frame}
  \frametitle{Problem H: Frontage Roads}

  \begin{itemize}
    \item First solve at 00:38 by \textbf{Skechers by Shawn} (Dhruv Patel and Adil Rasiyani)
    \item Answer $O(n^2)$ shortest-path queries for an unweighted graph with $n$ nodes
    \item Best algorithm choice is BFS, but Dijkstra's works as well
    \item \textbf{Potential pitfall}: BFS is $O(n)$, so running $O(n^2)$ BFSs is too slow
    \item BFS gives the shortest path to all other vertices
    \item Run $n$ BFSs up front, one for each vertex, then answer queries
    \item Total complexity: $O(n^2)$
  \end{itemize}
\end{frame}

\begin{frame}
  \frametitle{Problem H: Python Solution}
  \pyf{H}
\end{frame}

\begin{frame}
  \frametitle{Problem H: Python Solution (continued)}
  \pyf{H2}
\end{frame}

\begin{frame}
  \frametitle{Problem H: Python Solution (continued)}
  \pyf{H3}
\end{frame}
 % 0:42
\section{F}%
\label{sec:f}

\begin{frame}
  \frametitle{Problem F: Flummoxing Map}

  \begin{itemize}
    \item First solve at 00:58 by \textbf{0e4ef622} (Matthew Tran)
    \item Tricky problem, but lots of ways to solve it if you observe one of 2 things
    \item \textbf{Insight}: There are only 8 possible maps!
    \item \textbf{Insight}: This is just a linear transformation on the coordinates!
  \end{itemize}

\end{frame}

\begin{frame}
  \frametitle{Problem F: 8 Maps}
  \begin{itemize}
    \item We can uniquely identify a map by a rotation and whether it is mirrored
    \item Can think of this as a graph where the transformations are edges and there are 8 vertices
    \item You can convince yourself of this by trying various combinations of rotations and flips
    \item Or if you're into math you can prove it with group theory, I guess
    \item You can explicitly generate these maps, or just transform the coordinates depending on state
  \end{itemize}
\end{frame}

\begin{frame}
  \frametitle{Problem F: 8 Maps Solution}
  \pyf{F}
\end{frame}

\begin{frame}
  \frametitle{Problem F: 8 Maps Solution (continued)}
  \pyf{F2}
\end{frame}

\begin{frame}
  \frametitle{Problem F: Linear Transformation}
  \begin{itemize}
    \item If input coordinates are $i$, $j$, then mirroring changes them to $i$, $n-j+1$
    \item Rotating CCW replaces them with $j$, $n-i+1$
    \item Similar situation for flipping and rotating CW
    \item These are actually just 2x2 matrices
    \item Can just multiply a long string of 2x2 matrices to get our result
  \end{itemize}
\end{frame}
 % 0:58
\section{E}%
\label{sec:e}

\begin{frame}
  \frametitle{Problem E: Elfenheit}
  \begin{itemize}
    \item First solve at 01:01 by \textbf{ti123} (Hang Li)
    \item Given lists $a$, $b$, and $c$ (each of length $n$)
    \item Every iteration, we do $a[i] -= c[i]$ for all $i$, and we want to know how many elements in $a$
      are less then the corresponding element in $b$ after $k$ iterations
    \item There may be lots of iterations, and it will be too slow to do each subtraction one
      iteration at a time
  \end{itemize}
\end{frame}

\begin{frame}
  \frametitle{Problem E: C++ Solution}
  \cppf{E}
\end{frame}
\begin{frame}
  \frametitle{Problem E: C++ Solution (continued)}
  \cppf{E2}
\end{frame}
 % 1:11
\section{I}%
\label{sec:i}

\begin{frame}
  \frametitle{Problem I: Square Weed Whacker}

  \begin{itemize}
    \item First solve at 00:35 by \textbf{whatever} (Chinh Luu and Matthew Tran)
    \item We want the largest square such that every \texttt{.} can be covered without the square touching a \texttt{\#}
    \item One option is to binary search on square side length and use a prefix sum structure
    \item A cleaner approach uses a really nice insight: the side length is upper bounded by the shortest vertical or horizontal sequence of $\texttt{.}$
    \item A quick proof by contradiction (or a proof by AC) shows that we can attain this upper bound:
      \begin{itemize}
        \item Let $x$ be the shortest contiguous sequence of vertical or horizontal \texttt{.}
        \item Clearly, $x$ is an upper bound for the square's side, as otherwise we would definitely go out of bounds or hit a \texttt{\#}
        \item It remains to show that we can attain $x$. Indeed, if we couldn't, there must be a shorter sequence of \texttt{.} than $x$. But this is a contradiction!
      \end{itemize}
  \end{itemize}
\end{frame}

\begin{frame}
  \frametitle{Problem I: Python Solution}
  \pyf{I}
\end{frame} % :(
\section{J}%
 \setcounter{MaxMatrixCols}{20}
\label{sec:j}

\begin{frame}
  \frametitle{Problem J: Just Jingling}

  \begin{itemize}
    \item First solve at 00:17 by \textbf{whatever} (Chinh Luu and Matthew Tran)
    \item Given a lyric, how many times do we have to write it in a spiral to arrive back at the original lyric?
    \item \textbf{Insight:} we can reduce the problem to writing the integers $1\dots n$ in a spiral
      \begin{itemize}
        \item The fact that the input is simply the \textit{length} of the lyric should point to this observation
      \end{itemize}
  \end{itemize}
\end{frame}

\begin{frame}
  \frametitle{Problem J: Just Jingling (cont.)}

  \begin{itemize}
    \item \textbf{Insight:} we can reduce the problem to writing the integers $1\dots n$ in a spiral
    \item Let's try $n=13$ (sample case 3)
      \[
        \begin{array}{cccc}
            &    &   & 13\\
          5 & 4  & 3 & 12\\
          6 & 1 & 2 & 11\\
          7 & 8 & 9 & 10
        \end{array}
        \mapsto
        \begin{array}{cccc}
            &    &   & 10\\
          12 & 3  & 4 & 9\\
          6 & 13 & 5 & 8\\
          1 & 2 & 11 & 7
        \end{array}
        \mapsto
        \begin{array}{cccc}
            &    &   & 7\\
          9 & 4  & 3 & 11\\
          6 & 10 & 12 & 2\\
          13 & 5 & 8 & 1
        \end{array}
      \]
      \[
        \mapsto
        \begin{array}{cccc}
            &    &   & 1\\
          11 & 3  & 4 & 8\\
          6 & 7 & 9 & 5\\
          10 & 12 & 2 & 13
        \end{array}
        \mapsto \cdots \mapsto
        \begin{array}{cccc}
            &    &   & 13\\
          5 & 4  & 3 & 12\\
          6 & 1 & 2 & 11\\
          7 & 8 & 9 & 10
        \end{array}
      \]
  \end{itemize}
\end{frame}

\begin{frame}
  \frametitle{Problem J: Just Jingling (Cont.)}
  \begin{itemize}
    \item Let's relabel so that our original spiral looks like
      \[
      \begin{array}{cccc}
          &    &   & 1\\
        2 & 3  & 4 & 5\\
        6 & 7 & 8 & 9\\
        10 & 11 & 12 & 13
      \end{array}\mapsto
      \begin{array}{cccc}
            &    &   & 13\\
          5 & 4  & 3 & 12\\
          6 & 1 & 2 & 11\\
          7 & 8 & 9 & 10
        \end{array}
      \]
      Then the spiral induces a permutation on these elements, and we continue applying the permutation over
      and over again until it comes back to the original order.
    \item We can think of the permutation as a function $f$ from the integers $1\dots n$ to the
        integers $1\dots n$ that is one-to-one. It looks like
      \[ 
        \begin{pmatrix} 1 & 2 & 3 & 4 & 5 & 6 & 7 & 8 & 9 & 10 & 11 & 12 & 13\\
        13 & 5 & 4 & 3 & 12 & 6 & 1 & 2 & 11 & 7 & 8 & 9 & 10 \end{pmatrix}
      \]
  \end{itemize}
\end{frame}

\begin{frame}
  \frametitle{Problem J: Just Jingling (Cont.)}
  \begin{itemize}
    \item Then iterating the spiral process is equivalent to composing this function with itself multiple times.
      For example, the second spiral we write will be 
      \[ f\circ f = \]
      \[
      \begin{pmatrix} 1 & 2 & 3 & 4 & 5 & 6 & 7 & 8 & 9 & 10 & 11 & 12 & 13\\
      13 & 5 & 4 & 3 & 12 & 6 & 1 & 2 & 11 & 7 & 8 & 9 & 10 \end{pmatrix}
      \]
      \[
      \begin{pmatrix} 1 & 2 & 3 & 4 & 5 & 6 & 7 & 8 & 9 & 10 & 11 & 12 & 13\\
      13 & 5 & 4 & 3 & 12 & 6 & 1 & 2 & 11 & 7 & 8 & 9 & 10 \end{pmatrix}
    \]
      \[
      =
    \]
    \[
      \begin{pmatrix} 1 & 2 & 3 & 4 & 5 & 6 & 7 & 8 & 9 & 10 & 11 & 12 & 13\\
      10 & 12 & 3 & 4 & 9 & 6 & 13 & 5 & 8 & 1 & 2 & 11 & 7 \end{pmatrix}
    \]
  \end{itemize}
\end{frame}

\begin{frame}
  \frametitle{Problem J: Just Jingling (Cont.)}
  \begin{itemize}
    \item But notice that, in composing this permutation with itself many times, we will end up with 
      some elements cycling to each other. It will look like this:
      \begin{align*}
        f &= \begin{pmatrix} 1 & 2 & 3 & 4 & 5 & 6 & 7 & 8 & 9 & 10 & 11 & 12 & 13\\
      13 & 5 & 4 & 3 & 12 & 6 & 1 & 2 & 11 & 7 & 8 & 9 & 10 \end{pmatrix}\\
        &= (1,13,10,7)(2,5,12,9,11,8)(3,4)(6)
      \end{align*}
    \item These numbers will cycle, and we want to know when they are all \textit{simultaneously}
      at the beginning of the cycle, so we should just take the least common multiple of the cycle lengths!
  \end{itemize}
\end{frame}

\begin{frame}
  \frametitle{Problem J: Solution}
  \pyf{J}
\end{frame}
\begin{frame}
  \pyf{J2}
\end{frame}
\begin{frame}
  \pyf{J3}
\end{frame}
\begin{frame}
  \pyf{J4}
\end{frame}
 % :(
  
\end{document}
