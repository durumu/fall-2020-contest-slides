\section{X}%
\label{sec:X}

\begin{frame}
  \frametitle{Problem I: I Want Candy!}
  \begin{itemize}
    \item First solve at xx:yy by \textbf somebody
    \item At first glance, the problem might look familiar. Remember Knapsack problem?
    \item Let's ignore the special deals for now. We are left with a bunch of candies with happiness values and their prices. Let's find out what is the best deal we can get with $d$ money
    \item Let $dp[i][j]$ be the biggest happiness values we can get after we have examined the first $i$ candies and used $j$ money
%    \[ dp[i][j] = \begin{cases} dp[i-1][j-p[i]] + h[i], & \text{if we buy the $i^{th}$ candy}\\ dp[i-1][j], & \text{if we don't buy the $i^{th}$ candy} \end{cases}\]
    \item More precisely: $dp[i][j] = max(dp[i-1][j-p[i]]+h[i],dp[i-1][j])$
  \end{itemize}
\end{frame}

\begin{frame}
  \frametitle{Problem I: I Want Candy! (continued)}
   \begin{itemize}
    \item Now, let's take into account the special deals. Let's look at the following case:\\
        happiness: 5 7 9 1 2\\
        price: 8 3 2 10 1\\
        brand: 1 1 2 2 2\\
        brand price: 7 11\\
    \item If we sneak the special deals into the candy arrays like this:\\
        $[candy|brand]$ happiness: 5 7 \textbf{12} 9 1 2 \textbf{12}\\
        $[candy|brand]$ price: 8 3 \textbf{7} 2 10 1 \textbf{11}\\
        $[candy|brand]$ brand: 1 1 \textbf{1} 2 2 2 \textbf{2}\\
    \item Here, we can almost apply the Knapsack DP. Only 1 constraint left that we need to take care of. That is:
    \begin{itemize}
        \item If we buy a candy of a brand, we can't pay the special price to buy all candies of that brand. 
        \item If we pay the special price to buy all candies of a brand, we can't buy any individual candy in that brand.
    \end{itemize}
  \end{itemize}
\end{frame}

\begin{frame}
  \frametitle{Problem I: I Want Candy! (continued)}
  \begin{itemize}
    \item Let's modify the first case of our dp:
    \[dp[i][j] = dp[\boldsymbol{i-1}][j-p[i]] + h[i], \text{if we buy the $i^{th} [candy|brand]$}\]
    \item to:
    \[dp[i][j] = dp[\boldsymbol{i-k[i]}][j-p[i]] + h[i], & \text{if we buy the $i^{th} [candy|brand]$}\]
    $i-k[i]$ is the orderly-largest $[candy|brand]$ that can be bought together with the $i$-th $[candy|brand]$
    %\item \[ k[i] = \begin{cases}
    %  \text{\# of candies of $brand[i]$}, & \text{if a brand}\\
    %  1, & \text{if a candy}
    %\end{cases}
    %  \]
    \item With this, the constraint discussed in the previous slide is solved.
  \end{itemize}
\end{frame}

\begin{frame}
  \frametitle{Problem I: I Want Candy! (continued)}
  \begin{itemize}
    \item \textbf{The tricky part} of this problem is to place the brands and candies into an order that we can perform Knapsack-liked DP 
    \item Another "not-necessarily-hard" part of this problem is that you should know Knapsack problem :-)
    \item Although the idea is straightforward, coding up the part "sneak the brands into candy arrays" might be a bit annoying
  \end{itemize}
\end{frame}
