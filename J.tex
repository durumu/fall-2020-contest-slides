\section{J}%
 \setcounter{MaxMatrixCols}{20}
\label{sec:j}

\begin{frame}
  \frametitle{Problem J: Just Jingling}

  \begin{itemize}
    \item First solve at 00:17 by \textbf{whatever} (Chinh Luu and Matthew Tran)
    \item Given a lyric, how many times do we have to write it in a spiral to arrive back at the original lyric?
    \item Observation: we can reduce the problem to writing the integers $1\dots n$ in a spiral
      \begin{itemize}
        \item The fact that the input is simply the \textit{length} of the lyric should point to this observation
      \end{itemize}
  \end{itemize}
\end{frame}

\begin{frame}
  \frametitle{Problem J: Just Jingling (cont.)}

  \begin{itemize}
    \item Observation: we can reduce the problem to writing the integers $1\dots n$ in a spiral
    \item Let's try $n=13$ (sample case 3)
      \[
        \begin{array}{cccc}
            &    &   & 13\\
          5 & 4  & 3 & 12\\
          6 & 1 & 2 & 11\\
          7 & 8 & 9 & 10
        \end{array}
        \mapsto
        \begin{array}{cccc}
            &    &   & 10\\
          12 & 3  & 4 & 9\\
          6 & 13 & 5 & 8\\
          1 & 2 & 11 & 7
        \end{array}
        \mapsto
        \begin{array}{cccc}
            &    &   & 7\\
          9 & 4  & 3 & 11\\
          6 & 10 & 12 & 2\\
          13 & 5 & 8 & 1
        \end{array}
      \]
      \[
        \mapsto
        \begin{array}{cccc}
            &    &   & 1\\
          11 & 3  & 4 & 8\\
          6 & 7 & 9 & 5\\
          10 & 12 & 2 & 13
        \end{array}
        \mapsto \cdots \mapsto
        \begin{array}{cccc}
            &    &   & 13\\
          5 & 4  & 3 & 12\\
          6 & 1 & 2 & 11\\
          7 & 8 & 9 & 10
        \end{array}
      \]
  \end{itemize}
\end{frame}

\begin{frame}
  \frametitle{Problem J: Just Jingling (Cont.)}
  \begin{itemize}
    \item Let's relabel so that our original spiral looks like
      \[
      \begin{array}{cccc}
          &    &   & 1\\
        2 & 3  & 4 & 5\\
        6 & 7 & 8 & 9\\
        10 & 11 & 12 & 13
      \end{array}\mapsto
      \begin{array}{cccc}
            &    &   & 13\\
          5 & 4  & 3 & 12\\
          6 & 1 & 2 & 11\\
          7 & 8 & 9 & 10
        \end{array}
      \]
      Then the spiral induces a permutation on these elements, and we continue applying the permutation over
      and over again until it comes back to the original order.
    \item We can think of the permutation as a function $f$ from the integers $1\dots n$ to the
        integers $1\dots n$ that is one-to-one. It looks like
      \[ 
        \begin{pmatrix} 1 & 2 & 3 & 4 & 5 & 6 & 7 & 8 & 9 & 10 & 11 & 12 & 13\\
        13 & 5 & 4 & 3 & 12 & 6 & 1 & 2 & 11 & 7 & 8 & 9 & 10 \end{pmatrix}
      \]
  \end{itemize}
\end{frame}

\begin{frame}
  \frametitle{Problem J: Just Jingling (Cont.)}
  \begin{itemize}
    \item Then iterating the spiral process is equivalent to composing this function with itself multiple times.
      For example, the second spiral we write will be 
      \[ f\circ f = \]
      \[
      \begin{pmatrix} 1 & 2 & 3 & 4 & 5 & 6 & 7 & 8 & 9 & 10 & 11 & 12 & 13\\
      13 & 5 & 4 & 3 & 12 & 6 & 1 & 2 & 11 & 7 & 8 & 9 & 10 \end{pmatrix}
      \]
      \[
      \begin{pmatrix} 1 & 2 & 3 & 4 & 5 & 6 & 7 & 8 & 9 & 10 & 11 & 12 & 13\\
      13 & 5 & 4 & 3 & 12 & 6 & 1 & 2 & 11 & 7 & 8 & 9 & 10 \end{pmatrix}
    \]
      \[
      =
    \]
    \[
      \begin{pmatrix} 1 & 2 & 3 & 4 & 5 & 6 & 7 & 8 & 9 & 10 & 11 & 12 & 13\\
      10 & 12 & 3 & 4 & 9 & 6 & 13 & 5 & 8 & 1 & 2 & 11 & 7 \end{pmatrix}
    \]
  \end{itemize}
\end{frame}

\begin{frame}
  \frametitle{Problem J: Just Jingling (Cont.)}
  \begin{itemize}
    \item But notice that, in composing this permutation with itself many times, we will end up with 
      some elements cycling to each other. It will look like this:
      \begin{align*}
        f &= \begin{pmatrix} 1 & 2 & 3 & 4 & 5 & 6 & 7 & 8 & 9 & 10 & 11 & 12 & 13\\
      13 & 5 & 4 & 3 & 12 & 6 & 1 & 2 & 11 & 7 & 8 & 9 & 10 \end{pmatrix}\\
        &= (1,13,10,7)(2,5,12,9,11,8)(3,4)(6)
      \end{align*}
    \item These numbers will cycle, and we want to know when they are all \textit{simultaneously}
      at the beginning of the cycle, so we should just take the least common multiple of the cycle lengths!
  \end{itemize}
\end{frame}

\begin{frame}
  \frametitle{Problem J: Solution}
  \pyf{J}
\end{frame}
