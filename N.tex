\section{N}%
\label{sec:n}

\begin{frame}
  \frametitle{Problem N: Horses in the Back}

  \begin{itemize}
    \item First solve at 04:25 by \textbf{one man retirement home} (Chris Rech)
    \item We are given an integer $k$ and a function \[ f(x) = \begin{cases} x/k, & \text{if $k$ divides $x$}\\ x-1, & \text{otherwise} \end{cases}\]
      \item Let's denote by $len(x)$ the length of the sequence \[ x, f(x), f(f(x)), \dots, f(f(\cdots(x)\cdots)), 1 \]
    \item We want to find some $x$ such that $len(x)$ is maximal and $1\le x\le m$
  \end{itemize}
\end{frame}

\begin{frame}
  \frametitle{Problem N: Horses in the Back (continued)}
  \[ f(x) = \begin{cases} x/k, & \text{if $k$ divides $x$}\\ x-1, & \text{otherwise} \end{cases}\]
  \begin{itemize}
    \item Let's consider the case when $k=10$
      \begin{itemize}
        \item For $x=11$, the sequence is 11, 10, 1.
        \item For $x=111$, the sequence is 111, 110, 11, 10, 1.
        \item For $x=2111$, the sequence is 2111, 2110, 211, 210, 21, 20, 2, 1.
      \end{itemize}
    \item \textbf{Insight:} After some exploration, we see that when $k=10$, \[ len(x) = (\text{sum of digits in $x$}) + (\text{number of digits in $x$}) - 1. \]
    \item \textbf{Insight:} It leads us to the idea of considering the numbers in base $k$
  \end{itemize}
\end{frame}

\begin{frame}
  \frametitle{Problem N: Horses in the Back (continued)}
  \[ f(x) = \begin{cases} x/k, & \text{if $k$ divides $x$}\\ x-1, & \text{otherwise} \end{cases}\]
  \begin{itemize}
    \item So let's consider the case when $k=2$ (our favorite base)
      \begin{itemize}
        \item For $x=0b1011$, the sequence is 0b1011, 0b1010, 0b101, 0b100, 0b10, 0b1.
        \item For $x=0b1101011$, the sequence is 0b1101011, 0b1101010, 0b110101, 0b110100, 0b11010, 0b1101, 0b1100, 0b110, 0b11, 0b10, 0b1.
      \end{itemize}
    \item \textbf{Insight:} Let $x_k$ be $x$ written in base $k$. It turns out that our observation holds in general, \[ len(x) = (\text{sum of digits in $x_k$}) + (\text{number of digits in $x_k$}) - 1. \]
    \item \textbf{Insight:} For an intuition for this, we can just realize that this sequence exactly mimics the algorithm for converting between bases
  \end{itemize}
\end{frame}

\begin{frame}
  \frametitle{Problem N: Horses in the Back (continued)}
  \begin{itemize}
    \item We have transformed the problem, but now: how to solve the original?
    \item Consider $m$ (the maximum allowed value) in base $k$
    \item Either $m$ is the answer, or the answer is a prefix of $m$ in base $k$ followed by a bunch of $k-1$ digits
    \item As an example, we go back to $k=10$ since it is our most familiar base
      \begin{itemize}
        \item Let's say $m=20$, then our answer is $x=19$
      \end{itemize}
    \item \textbf{Potential pitfall:} Consider the case $k=3$, $m=15$
      \begin{itemize}
        \item In base 3, we have $m=120$. You may try $x_3=120$ and $x_3=22$, but the best solution is $x_3=112$, whose length is $(2+1+1)+3-1=6$.
      \end{itemize}
    \item The hard part of the problem is thinking, and you are probably getting hungry, so we leave the final solution to you :-)
  \end{itemize}
\end{frame}
