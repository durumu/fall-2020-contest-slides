\documentclass{beamer}
\usepackage{listings}
\usepackage{textcomp}
\usepackage{amsmath}

% Listings
    \definecolor{dkgreen}{rgb}{0,0.6,0}
    \definecolor{gray}{rgb}{0.5,0.5,0.5}
    \definecolor{mauve}{rgb}{0.58,0,0.82}

    \lstloadlanguages{[ISO]C++, Python}

    \lstset{frame=tb,
      upquote=true,
      language=[ISO]C++,
      aboveskip=2mm,
      belowskip=2mm,
      showstringspaces=false,
      columns=flexible,
      basicstyle={\small\ttfamily},
      numbers=none, % line numbers -- change to "left" if you want them
      numberstyle=\tiny\color{gray},
      keywordstyle=\bfseries\color{blue},
      commentstyle=\itshape\color{dkgreen},
      stringstyle=\color{mauve},
      breaklines=true,
      breakatwhitespace=true,
      tabsize=2
    }

    % C++ environment
    \lstnewenvironment{cpp} {
      \lstset{
        language=[ISO]C++,
        tabsize=2
      }
    } {}

    % Python environment
    \lstnewenvironment{py} {
      \lstset{
        language=Python,
        tabsize=4
        }
    } {}

\newcommand{\cppf}[1]{\lstinputlisting{#1.cpp}}
\newcommand{\pyf}[1]{\lstinputlisting[language=Python, upquote=true]{#1.py}}

\usetheme{Madrid}

\title{Editorial}
\subtitle{Texas A\&M 2020 Spring Contest}
\date{\today}

\begin{document}

\section{A}%
\label{sec:a}

\begin{frame}
  \frametitle{Problem A: Are You Ready?}

  \begin{itemize}
    \item First solve at 00:00 by \textbf{alab11} (Alex Labbane)
    \item A little introduction to today's theme!
  \end{itemize}

  \pyf{A}
\end{frame}
 % 0:01
\section{M}%
\label{sec:m}

\begin{frame}
  \frametitle{Problem M: Family Photo}

  \begin{itemize}
    \item First solve at 00:05 by \textbf{whatever} (Chinn Luu and Matthew Tran)
    \item Line up your family in non-decreasing order
    \item \textbf{Insight}: You can only control the orderings of the family members of equal height
    \item If there are $k$ people with height $h$, then there are $k!$ ways to order people of that height
    \item Multiply $k!$ together for each height and you will have the solution
    \item \textbf{Potential pitfall}: You have to take your solution mod 998244353 at intermediate steps as well
    \item Otherwise, your solution will TLE (if Python) or overflow (if C++/Java)
    \item Total complexity: $O(n)$ using hashmaps
  \end{itemize}
\end{frame}

\begin{frame}
  \frametitle{Problem M: Solution}
  \pyf{M}
\end{frame}
 % 0:05
\section{B}%
\label{sec:B}

\begin{frame}
  \frametitle{Problem B: Build a Snowman}
  \begin{itemize}
    \item First solve at 00:07 by \textbf{i\_am\_a\_business\_major} (Logan Baker)
    \item \textbf{Observation}: any values in between the first and the last values in the build array does not contribute to the sturdiness factor at all.\\
    $\frac{a_2}{a_1} * \frac{a_3}{a_2} * \frac{a_4}{a_3} * ... * \frac{a_{n-1}}{a_{n-2}} * \frac{a_n}{a_{n-1}} = \frac{a_n}{a_1}$
    \item If we choose the maximum radius as $a_n$ and the minimum radius as $a_1$, the sturdiness factor will be maximized.
    \item \textbf{Potential pitfall}: The answer has to be precise to $10^-6$. Your answer could be considered wrong if you don't print enough decimal digits. For example: $2.62$ is interpreted as $2.6200000$ while the correct answer can be $2.6200011$.
  \end{itemize}
\end{frame}


\begin{frame}
	\frametitle{Problem B: C++ solution}
	\cppf{B}
\end{frame}
 % 0:08
\section{C}%
\label{sec:c}

\begin{frame}
  \frametitle{Problem C: Cold War}

  \begin{itemize}
    \item First solve at 00:27 by \textbf{Jonathan1234} (Naixin Zong)
    \item Implementation problem: figure out if you can make it out of a duel without getting shot
    \item One possible solution: try shooting everyone and see what happens 
    \item Complexity: $O(n^2)$
    \item Another possible solution: order everyone by reaction time, then step through until your turn
    \item Complexity: $O(n \log n)$
  \end{itemize}

\end{frame}

\begin{frame}
  \frametitle{Problem C: $O(n^2)$ try-everyone solution}
  \pyf{Cbrute}
\end{frame}

\begin{frame}
  \frametitle{Problem C: $O(n^2)$ try-everyone solution (continued)}
  \pyf{Cbrute2}
\end{frame}

\begin{frame}
  \frametitle{Problem C: $O(n \log n)$ ordering solution}
  \pyf{Cfast}
\end{frame}
 % 0:14
\section{J}%
\label{sec:j}

\begin{frame}
  \frametitle{Problem J: History Lesson}

  \begin{itemize}
    \item First solve at 00:17 by \textbf{whatever} (Chinh Luu and Matthew Tran)
    \item Straightforward implementation problem
    \item Just use a hashmap mapping strings to lists
    \item \texttt{unordered\_map<string, vector<string>>} in C++
    \item \texttt{collections.defaultdict(list)} in Python
    \item Some teams struggled with sorting the list alphabetically
  \end{itemize}
\end{frame}

\begin{frame}
  \frametitle{Problem J: Solution}
  \pyf{J}
\end{frame}
 % 0:17
\section{D}%
\label{sec:d}

\begin{frame}
  \frametitle{Problem D: Dubious Logs}

  \begin{itemize}
    \item First solve at 00:22 by \textbf{whatever} (Chinh Luu and Matthew Tran)
    \item We can reduce the problem to the following: given a list $a$ and a list $l$,
      match each element of $l$ to an element of $a$ that is less than or equal to it.
    \item \textbf{Observation:} it is never worse to assign an element of $l$ closest
      element of $a$ less than or equal to it that we have not already assigned, so we
      can just sort both lists and try this assignment.
  \end{itemize}
  \pyf{D}
\end{frame}
 % 0:22
%\section{I}%
\label{sec:i}

\begin{frame}
  \frametitle{Problem I: Square Weed Whacker}

  \begin{itemize}
    \item First solve at 00:35 by \textbf{whatever} (Chinh Luu and Matthew Tran)
    \item We want the largest square such that every \texttt{.} can be covered without the square touching a \texttt{\#}
    \item One option is to binary search on square side length and use a prefix sum structure
    \item A cleaner approach uses a really nice insight: the side length is upper bounded by the shortest vertical or horizontal sequence of $\texttt{.}$
    \item A quick proof by contradiction (or a proof by AC) shows that we can attain this upper bound:
      \begin{itemize}
        \item Let $x$ be the shortest contiguous sequence of vertical or horizontal \texttt{.}
        \item Clearly, $x$ is an upper bound for the square's side, as otherwise we would definitely go out of bounds or hit a \texttt{\#}
        \item It remains to show that we can attain $x$. Indeed, if we couldn't, there must be a shorter sequence of \texttt{.} than $x$. But this is a contradiction!
      \end{itemize}
  \end{itemize}
\end{frame}

\begin{frame}
  \frametitle{Problem I: Python Solution}
  \pyf{I}
\end{frame} % 0:35
\section{H}%
\label{sec:h}

\begin{frame}
  \frametitle{Problem H: Help the Carolers}

  \begin{itemize}
    \item First solve at 00:38 by \textbf{Name Here}
    \item Determine if there is a word or group of words present in both the end of the first phrase and start of the second
    \item Iterate backwards over first phrase checking for equality
    \item \textbf{Potential pitfall}: Case where the second phrase is non-ending substring of the first
    \item Total complexity: $O(n^2)$
  \end{itemize}
\end{frame}

\begin{frame}
  \frametitle{Problem H: Python Solution}
  \pyf{H}
\end{frame}
 % 0:38
\section{G}%
\label{sec:g}

\begin{frame}
  \frametitle{Problem G: Size Constraints}

  \begin{itemize}
    \item First solve at 00:38 by \textbf{backslash7} (Matthew Davis, Nicholas Glaze, and Travis Baylor)
    \item We want to change the minimum number of elements in $a$ such that $b$ never occurs as a continuous subsegment
    \item \textbf{Insight:} since there are so many numbers that are allowed as sizes, we can get rid of any occurrence
    \item Let's go left to right until we find the first occurrence of $b$. Clearly, we must change something.
      \begin{itemize}
        \item We can greedily choose to change the last element of the occurrence. It still gets rid of this occurrence, and ``has the most potential''
          to remove future occurrences
      \end{itemize}
    \item \textbf{Potential pitfall:} $b$ can be longer than $a$, in which case the answer is always 0
  \end{itemize}
\end{frame}

\begin{frame}
  \frametitle{Problem G: Python Solution}
  \pyf{G}
\end{frame}
 % 0:38
\section{L}%
\label{sec:l}

\begin{frame}
  \frametitle{Problem L: Obstacle Course}

  \begin{itemize}
    \item First solve at 00:57 by \textbf{yeetmoney} (Yamen Almasalmeh and Reggie Frank)
    \item Want to know the number of distinct paths through a string
    \item We can either go to the next position, or jump forward to the next
      same character \textit{at most} $k$ times
    \item \textbf{Insight:} this problem lends itself to recursive thinking:
      \begin{itemize}
        \item ``If I am at index $i$, where could I have come from?''
        \item The answer is that you either came from the previous index, or the last index which has the same character
        \item Denote by $f(i, j)$ the number of paths ending at index $i$ which jump forward exactly $j$ times
        \item Denote by $prev(i)$ the previous location of the character at $i$ (if it exists)
      \end{itemize}
  \end{itemize}

\end{frame}

\begin{frame}
  \frametitle{Problem L: Obstacle Course (continued)}
  \begin{itemize}
    \item Denote by $f(i, j)$ the number of paths ending at index $i$ which jump forward exactly $j$ times
    \item Denote by $prev(i)$ the previous location of the character at $i$ (if it exists)
    \item \[
        f(i, j) = \begin{cases}
          f(i-1, j), & $j = 0$\\
          f(i-1, j), & \text{$prev(i)=i-1$ or DNE}\\
          f(i-1, j) + f(prev(i), j-1), & \text{otherwise.}
        \end{cases}
      \]
    \item We also have the base case: $f(0, j) = [j=0]$
    \item Can solve this recurrence with dynamic programming
    \item \textbf{Potential pitfall:} Must handle the base case, and also the special cases of the recurrence
  \end{itemize}
\end{frame}

\begin{frame}
  \frametitle{Problem L: Python Solution, Top-Down DP}
  \pyf{Ltop}
\end{frame}

\begin{frame}
  \frametitle{Problem L: Python Solution, Top-Down DP (continued)}
  \pyf{Ltop1}
\end{frame}

\begin{frame}
  \frametitle{Problem L: Python Solution, Bottom-Up DP}
  \pyf{Lbot}
\end{frame}
 % 0:57
\section{F}%
\label{sec:f}

\begin{frame}
  \frametitle{Problem F: Flummoxing Map}

  \begin{itemize}
    \item First solve at 00:58 by \textbf{0e4ef622} (Matthew Tran)
    \item Tricky problem, but lots of ways to solve it if you observe one of 2 things
    \item \textbf{Observation}: There are only 8 possible maps!
    \item \textbf{Observation}: This is just a linear transformation on the coordinates!
  \end{itemize}

\end{frame}

\begin{frame}
  \frametitle{Problem F: 8 Maps}
  \begin{itemize}
    \item We can uniquely identify a map by a rotation and whether it is mirrored
    \item Can think of this as a graph where the transformations are edges and there are 8 vertices
    \item You can convince yourself of this by trying various combinations of rotations and flips
    \item Or if you're into math you can prove it with group theory, I guess
    \item You can explicitly generate these maps, or just transform the coordinates depending on state
  \end{itemize}
\end{frame}

\begin{frame}
  \frametitle{Problem F: 8 Maps Solution}
  \pyf{F}
\end{frame}

\begin{frame}
  \frametitle{Problem F: Linear Transformation}
  \begin{itemize}
    \item If input coordinates are $i$, $j$, then mirroring changes them to $i$, $n-j+1$
    \item Rotating CCW replaces them with $j$, $n-i+1$
    \item Similar situation for flipping and rotating CW
    \item These are actually just 2x2 matrices
    \item Can just multiply a long string of 2x2 matrices to get our result
  \end{itemize}
\end{frame}

\begin{frame}
  \frametitle{Problem F: Linear Transformation Solution}
  \cppf{F}
\end{frame}
 % 0:58
\section{E}%
\label{sec:e}

\begin{frame}
  \frametitle{Problem E: Chili Cook-off}
  \begin{itemize}
    \item First solve at 01:01 by \textbf{ti123} (Hang Li)
    \item The event space consists of every possible combination of trucks starting. There are $2^n$, since each truck either starts or doesn't
    \item Notice that $n \le 10$, so we can use brute force
    \item You can use recursion, but an easier way is to recognize that each bitmask of length $n$
      corresponds to a unique event, where person $i$ has a good truck day if bit $i$ is set
  \end{itemize}
\end{frame}
\begin{frame}
  \begin{itemize}
    \item So let's loop over all bitmasks of length $n$, and for each, we can calculate the probability of this event as \[ \left( \prod_{\text{bit $i$ is 1}} p[i] \right) \left( \prod_{\text{bit $i$ is 0}} (1-p[i]) \right). \]
    \item Once we have the probability, we determine what place we would get if this event occurs, and add this probability to the output for that place
    \item Bonus: can you solve it in $O(n^2)$?
  \end{itemize}
\end{frame}

\begin{frame}
  \frametitle{Problem E: C++ Solution}
  \cppf{E}
\end{frame}
\begin{frame}
  \frametitle{Problem E: C++ Solution (continued)}
  \cppf{E2}
\end{frame}
 % 1:01
\section{N}%
\label{sec:n}

\begin{frame}
  \frametitle{Problem N: Horses in the Back}

  \begin{itemize}
    \item First solve at 04:25 by \textbf{one man retirement home} (Chris Rech)
    \item We are given an integer $k$ and a function \[ f(x) = \begin{cases} x/k, & \text{if $k$ divides $x$}\\ x-1, & \text{otherwise} \end{cases}\]
      \item Let's denote by $len(x)$ the length of the sequence \[ x, f(x), f(f(x)), \dots, f(f(\cdots(x)\cdots)), 1 \]
    \item We want to find some $x$ such that $len(x)$ is maximal and $1\le x\le m$
  \end{itemize}
\end{frame}

\begin{frame}
  \frametitle{Problem N: Horses in the Back (continued)}
  \[ f(x) = \begin{cases} x/k, & \text{if $k$ divides $x$}\\ x-1, & \text{otherwise} \end{cases}\]
  \begin{itemize}
    \item Let's consider the case when $k=10$
      \begin{itemize}
        \item For $x=11$, the sequence is 11, 10, 1.
        \item For $x=111$, the sequence is 111, 110, 11, 10, 1.
        \item For $x=2111$, the sequence is 2111, 2110, 211, 210, 21, 20, 2, 1.
      \end{itemize}
    \item \textbf{Insight:} After some exploration, we see that when $k=10$, \[ len(x) = (\text{sum of digits in $x$}) + (\text{number of digits in $x$}) - 1. \]
    \item \textbf{Insight:} It leads us to the idea of considering the numbers in base $k$
  \end{itemize}
\end{frame}

\begin{frame}
  \frametitle{Problem N: Horses in the Back (continued)}
  \[ f(x) = \begin{cases} x/k, & \text{if $k$ divides $x$}\\ x-1, & \text{otherwise} \end{cases}\]
  \begin{itemize}
    \item So let's consider the case when $k=2$ (our favorite base)
      \begin{itemize}
        \item For $x=0b1011$, the sequence is 0b1011, 0b1010, 0b101, 0b100, 0b10, 0b1.
        \item For $x=0b1101011$, the sequence is 0b1101011, 0b1101010, 0b110101, 0b110100, 0b11010, 0b1101, 0b1100, 0b110, 0b11, 0b10, 0b1.
      \end{itemize}
    \item \textbf{Insight:} Let $x_k$ be $x$ written in base $k$. It turns out that our observation holds in general, \[ len(x) = (\text{sum of digits in $x_k$}) + (\text{number of digits in $x_k$}) - 1. \]
    \item \textbf{Insight:} For an intuition for this, we can just realize that this sequence exactly mimics the algorithm for converting between bases
  \end{itemize}
\end{frame}

\begin{frame}
  \frametitle{Problem N: Horses in the Back (continued)}
  \begin{itemize}
    \item We have transformed the problem, but now: how to solve the original?
    \item Consider $m$ (the maximum allowed value) in base $k$
    \item Either $m$ is the answer, or the answer is a prefix of $m$ in base $k$ followed by a bunch of $k-1$ digits
    \item As an example, we go back to $k=10$ since it is our most familiar base
      \begin{itemize}
        \item Let's say $m=20$, then our answer is $x=19$
      \end{itemize}
    \item \textbf{Potential pitfall:} Consider the case $k=3$, $m=15$
      \begin{itemize}
        \item In base 3, we have $m=120$. You may try $x_3=120$ and $x_3=22$, but the best solution is $x_3=112$, whose length is $(2+1+1)+3-1=6$.
      \end{itemize}
    \item The hard part of the problem is thinking, and you are probably getting hungry, so we leave the final solution to you :-)
  \end{itemize}
\end{frame}
 % unsolved
\section{K}%
\label{sec:k}

\begin{frame}
  \frametitle{Problem K: Trick Shot}

  \begin{itemize}
    \item Nobody really attempted this one
    \item Find out whether there exist four lines that contain every point. Let's view this in a different way
    \item We assume the points were placed on one of four different lines and try to find these lines (if we can't, then we say they don't exist)
    \item We are tempted to use brute-force, but this is too slow
    \item \textbf{Insight} \begin{itemize}
      \item Suppose we have found $k+1$ points that are all collinear (check collinearity using cross product)
      \item Then, if there are $k$ lines we can draw that intersect all the points, one will be that line
      \item If we can't find those $k+1$ points and $n > k^2$ the problem is impossible
      \item If we can find the $k+1$ points, we remove all the points on the line we found, and recurse with $k = k-1$
    \end{itemize}
    \item If $n \le k^2$, since that is only 16, we can do a brute force and it will be fast enough
  \end{itemize}

\end{frame}

\begin{frame}
  \frametitle{Problem K: Monte Carlo}

  \begin{itemize}
    \item \textbf{Insight}: If we pick two points at random, there is a pretty high ($\ge \frac{1}{4}$) chance they will lie on the same line
    \item Then, if we remove all the points on that line and pick two more, there is $\ge \frac{1}{3}$ chance they will lie on another line
    \item Then, if we remove all the points on \textit{that} line and pick two more, there is $\ge \frac{1}{2}$ chance they will lie on another line
    \item Worst case scenario, if a solution exists, there is a $\frac{1}{24}$ chance we find it. Solution is verified in $O(n)$ time
    \item We can run at least 1,000 tests during the time limit of 10 seconds, using C++ (more like 5,000 but let's be conservative)
    \item Chance of finding a solution, if it exists, is $(\frac{23}{24})^{1000} \approx 3 \times 10^{-19}$
    \item Therefore, our solution is basically guaranteed to work (have you ever been 99.99999999999999997\% sure about \textit{anything}?)
  \end{itemize}
  
\end{frame}
 % unsolved
  
\end{document}
