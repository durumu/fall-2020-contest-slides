\section{E}%
\label{sec:e}

\begin{frame}
  \frametitle{Problem E: Elfenheit}
  \begin{itemize}
    \item First solve at 01:11 by \textbf{tbolton2000} (Trevor Bolton)
    \item Given lists $a$, $b$, and $c$ (each of length $n$)
    \item Every iteration, we do $a[i] = a[i] - c[i]$ for all $i$, and we want to know how many
      positions $i$ exist such that $a[i] < c[i]$ after $k$ iterations
    \item \textbf{Insight:} instead of recomputing the answer every time, we can first precompute how long each
      neighborhood in $a$ will take to pass the corresponding temperature in $c$
    \item Let's create a new array $d[i] = (a[i] - c[i]) / b[i]$. Then $d[i]$ is the time in
      milliseconds that it will take for the $i$-th neighborhood to begin working (in math; in code
      we might want to take the ceiling of this)
    \item If we sort this array, we can binary search for the time we are looking for and then just see
      what index we end up at -- this is our answer!
  \end{itemize}
\end{frame}

\begin{frame}
  \frametitle{Problem E: Python Solution}
  \pyf{E}
\end{frame}
